\documentclass{article}
\usepackage[utf8]{inputenc}
\usepackage[french]{babel}
\usepackage[T1]{fontenc}
\usepackage{graphicx}
\usepackage{hyperref}
\usepackage{caption}
\usepackage{float}
\usepackage[linesnumbered,ruled]{algorithm2e}
\usepackage[dvipsnames]{xcolor}
\usepackage{tabto}
\newenvironment{code}[1]
    {\begin{center}
    \textbf{#1\\[1ex]}
    \begin{tabular}{|p{0.9\textwidth}|}
    \hline\\
    }
    { 
    \\\\\hline
    \end{tabular} 
    \end{center}
    }
    
\newcommand{\size}[2]{{\fontsize{#1}{0}\selectfont#2}}
\newenvironment{sizepar}[2]
 {\par\fontsize{#1}{#2}\selectfont}
 {\par}

\begin{document}
\begin{titlepage}
	\parindent0pt
	\hskip.25cm \vrule width.8pt \hskip.25cm
	\vskip0pt plus 1fil
	\fbox{%
		\begin{minipage}{\dimexpr\linewidth-2\fboxsep-2\fboxrule}
			\centering\Large\bfseries
			\vskip1cm
			Autres Paradigmes
			\vskip1cm \kern0pt
		\end{minipage}%
	}%
	\vskip0.5cm
	\centerline{\textbf{DM Haskell}}
	\vskip0pt plus 1fil
	\vskip0.25cm
	\centerline{HUET Bryan\\ SAHIN Tolga}
	\leavevmode
	\vskip0pt plus 2fil
	\hfill\bfseries{Année universitaire 2020}\hfill\null
\end{titlepage}

\newpage

\thispagestyle{empty}
\tableofcontents
\newpage


\section{Mise sous forme clausale d’une formule du calcul propositionnel (Partie 1)}
Pour la bonne compréhension du dm, nous rappellerons à quel opérateurs logique correspond chaque caractère:\\
Soit g et d deux formules,\\
\~g :  Non g \\
g \& d : g Et d \\
g v d : g Ou d \\
g => d : g Implique d\\
g <=> d : g Équivaut d\\


\subsection{Visualiser une formule : Question 1}	
\begin{code}{visuFormule}
    visuFormule :: Formule -> String \\
    \\
    visuFormule (Var p) = p \\
    visuFormule (Non f) = "\~" ++ visuFormule f\\
    visuFormule (Et g d) = "(" ++ (visuFormule g) ++ " \& " \\
                            ++ (visuFormule d) ++ ")" \\
    visuFormule (Ou g d) =  "(" ++ (visuFormule g) ++ " v " \\
                            ++ (visuFormule d) ++ ")" \\
    visuFormule (Imp g d) = "(" ++ (visuFormule g) ++ " => " \\ 
                            ++ (visuFormule d) ++ ")" \\
    visuFormule (Equi g d) ="(" ++ (visuFormule g) ++ " <=> " \\
                        ++ (visuFormule d) ++ ")" 
\end{code}

\subsection{Faire disparaître des opérateurs : Question 2 \& 3}
Soit g et d deux formule, \\
- Il est possible de remplacer l'opérateur d'implication dans g => d par : \\
(g => d) = (\~g v d) \\
- Il est possible de remplacer l'opérateur d'équivalence dans g <=> d par : \\
(g <=> d) = ((g => d) \& (d => g))
\newpage
\begin{code}{elimine}
    elimine :: Formule -> Formule\\
    \\
    elimine (Var p) = (Var p)\\
    elimine (Non f) = (Non (elimine f))\\
    elimine (Et g d) = (Et (elimine g) (elimine d))\\
    elimine (Ou g d) =(Ou (elimine g) (elimine d))\\
    elimine (Imp g d) = (Ou (Non (elimine g)) (elimine d))\\
    elimine (Equi g d) = (Et (elimine (Imp (elimine g) (elimine d)))\\ (elimine (Imp (elimine d) (elimine g))))

\end{code}

\subsection{Amener les négations devant les littéraux positifs : Question 4,5,6}
Soit g, d et f des formules,\\
Grâce à la loi de la double négation, Non(Non f) devient f.\\
Les deux lois de Morgan sont les suivantes :\\
- Non (g v d) <=> (\~g) \& (\~d)\\
- Non (g \& d) <=> (\~g) v (\~d)
\\
\\
disNon doit appliquer les lois de Morgan et de la double négation.
\begin{code}{ameneNon, disNon}
    ameneNon, disNon :: Formule -> Formule\\
    \\
    ameneNon (Var p) = (Var p)\\
    ameneNon (Non f) = disNon f\\
    ameneNon (Et g d) = (Et (ameneNon g) (ameneNon d))\\
    ameneNon (Ou g d) = (Ou (ameneNon g) (ameneNon d))\\
    \\
    disNon (Var p) = (Non (Var p))\\
    disNon (Non f) = (f)\\
    disNon (Et g d) = (Ou (disNon g) (disNon d))\\
    disNon (Ou g d) = (Et (disNon g) (disNon d))\\
\end{code}
\newpage

\subsection{Faire apparaître une conjonction de clauses : Question 7 \& 8}
\begin{code}{developper}
    developper :: Formule -> Formule -> Formule\\
    \\
    developper (Et g d) x = (Et (Ou x g) (Ou x d))\\
    developper x (Et g d) = (Et (Ou x g) (Ou x d))\\
    developper x y = (Ou x y)
\end{code}
\begin{code}{formeClausale}
    formeClausale :: Formule -> Formule\\
    \\
    formeClausale f = normalise (ameneNon (elimine f))
\end{code}
\newpage

\section{Résolvante et principe de résolution (Partie 2)}

\subsection{Transformer une Formule en une FormuleBis : Question 9}
\begin{code}{etToListe}
    etToListe :: Formule -> FormuleBis\\
    \\
    etToListe (Et g d) = (ouToListe g) : (etToListe d)\\
    etToListe f = [ouToListe f]
\end{code}
\begin{code}{ouToListe}
    ouToListe :: Formule -> Clause\\
    \\
    ouToListe (Ou g d) = (ouToListe g )++(ouToListe d)\\
    ouToListe f = [f]
\end{code}
\subsection{Résolvante de deux clauses : Question 10,11,12}
\begin{code}{neg}
    neg :: Formule -> Formule\\
    \\
    neg (Non f) = f\\
    neg f = (Non f)
\end{code}
\begin{code}{sontLiees}
    sontLiees :: Clause -> Clause -> Bool\\
    \\
    sontLiees [] \_ = False\\
    sontLiees (x:xs) ys = ((neg x) `elem` ys) || (sontLiees xs ys)
\end{code}
\newpage
\begin{code}{resolvante}
    resolvante :: Clause -> Clause -> Clause\\
    \\
    resolvante [] ys = []\\
    resolvante (x:xs) (y:ys)\\
    \tabto{5mm}| ((neg x) `elem` (y:ys))== True =  xs ++ (delete (neg x) (y:ys))\\
    \tabto{5mm}| ((neg y) `elem` (x:xs))== True =  ys ++ (delete (neg y) (x:xs))\\
    \tabto{5mm}| otherwise = [x] ++ [y] ++ resolvante xs ys
    
        
\end{code}
\section{Application : les sorites de Lewis Carroll}

\end{document}